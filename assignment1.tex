\documentclass{article}
\usepackage{xcolor}
\usepackage{amsmath}
\usepackage{graphicx}
\title{\textbf{THE 18.821 MATHEMATICS PROJECT LAB REPORT
	[REPLACE THIS WITH YOUR OWN SHORT
	DESCRIPTIVE TITLE!]}}
\author{X.BURPS,P.GURPS}
\begin{document}
	\maketitle
		ABSTRACT. This is a LATEX template for 18.821, which you can\newline
		use for your own reports.
		\centering
	  \section{INTRODUCTION}
	  \large{This brief document shows some examples of the use of LATEX and
	  indicates some special features of the Math Lab report style. The
	  \textcolor{blue}{course website} contains links to several LATEX manuals.\newline
	 \hspace{0.5cm}End the introduction by describing the contents of the paper sec­ 
	  tion by section, and which team member(s) wrote each of them. For
	  instance, Section \textcolor{blue}{6} discusses referencing, and is written by P. Gurps.}
 \section{LATEX EXAMPLES}
 \large{Here are some ways of producing mathematical symbols. Some are \newline
 pre-defined either in LATEX or in the AMS package which this document \newline loads.For instance,sums and integrals,$\sum_{i=1}^{n}$1 =n,$\int_{0}^{n}$ x dx = n2/2.\newline We've defined a few other symbols at the start of the document,for\newline instance N,Q,Z,R. You can make marginal notes for yourself or your co-authors like this:\newline
 If you want to typeset equations, there are many choices, with or without numbering:}\newline
 
 $\int_{0}^{1}$ x dx = 1/2,\\ \vspace{1cm} or \newline
 \vspace{1cm}
 $\sum_{i=1}^{\infty}$ i = -$\frac{1}{12}$ \\ 
 or \newline 
 \hspace{1cm}1 -1 + 1 - ··· = $\frac{1}{2}$.
 \newline
 \begin{figure}[h]
        \centering
        \includegraphics[width=1.0\textwidth,height=8cm]{/home/rgukt-basar/Pictures/Screenshots/graph.jpg}
        \caption{My first.pdf figure}
        \label{fig:first}
 \end{figure}
           \newline 
            \vspace{1cm} 
            \newline X. BURPS, P. GURPS
            \newline
           \large{If you want a number for an equation, do it like this: \newline 
           	(1)      $\lim_{n\to\infty}$ $\sum_{k=1}^{n}$$\frac{1}{k^2}$ = $\frac{\pi}{6}$.\newline
           	This can then be referred to as (1), which is much easier than keeping \newline
           	track of numbers by hand. To group several equations, aligning on the \newline
           	= sign, do it like this: \newline
           	 $x_1+2x_2+3x_3 = 7$ \newline
           	          y = mx+c \newline
           	            =4x-9. \newline
           	    You can easily embed hyperlinks into the output .pdf document: \newline
           	            \textcolor{blue}{click here }for example.
 \section{IMAGES} 
\large{Figure 1 is an example of a .pdf image put into a floating environ- \newline
 ment, which means LaTeX will draw it wherever there’s enough space 
 left in your manuscript. Look at the .tex original to see how to insert 
 a figure like this.}
\section{THEOREMS AND SUCH}
\large{An example of a “conjecture environment” is given below, in Con- \newline
jecture 4.1. Theorems, lemmas, propositions, definitions, and such \newline all 
use the same command with the appropriate name changed. In fact,\newline
       THE 18.821 REPORT \newline
if you look at the top of this .tex file, you can see where we’ve defined 
these environments.\newline
\textbf{Conjecture 4.1} (Vaught’s Conjecture). Let T be a countable com­plete 
then it has countably many countable models. \newline
\textbf{Theorem 4.2.} When it rains it pours.
Proof. Well, yes.}

\section{FILETYPES USED BY LATEX}
\large{You will write your text as a .tex file using any text editor (though \newline
WYSIWYG editors are troublesome). Traditionally one then runs 
LATEX and obtains a .dvi file, which can be viewed on the screen using a 
dvi viewer. To include images, and then prepare the file for printing or 
submission, one typically translates the .dvi into either .ps (Postscript)
or .pdf (Adobe PDF).
Your report will be submitted as a .pdf document. The \textbf{pdflatex }
command produces a .pdf file directly from a .tex file. This command 
works well with included .pdf files, but does not handle .eps files. 
An .eps file can be converted to a .pdf file by viewing it and saving 
as a .pdf file, or by \textbf{ps2pdf filename.eps}, which produces 
\textbf{filename.pdf.} Under MikTeX with WinEdt, all necessary commands 
will appear under “Accessories” in the WinEdt menu. 
Finally, Matlab can be made to produce .eps files by typing 
\textbf{print -deps filename} 
at the prompt.}
\section{QUOTING SOURCES}
\large{In your work, keep notes of the literature you’ve used, including \newline
websites. Cite the references you use; failure to do so constitutes pla- 
giarism. Every bibliography item should be referenced somewhere in 
the paper. Quote as precisely as possible: [1, pages 76–78] rather than 
[1]. [2] was a useful background reference, too.} \newline
\textbf{REFERENCES} \newline
\large{[1] Gurps, P., Care and feeding of maths professors. Cambridge Univ. Press, 2008. \newline
[2] Burps, X. Terrors and errors of project lab. Journal of Wildlife and Conservation 21 (2008), 112–134.} \newline
\textbf{APPENDIX} \newline
\large{Appendices are useful for putting in code or data.}\newline

MIT OpenCourseWare \newline
\textcolor{blue}{http://ocw.mit.edu} \vspace{0.5cm} \newline
\bf{18.821 Project Laboratory in Mathematics \newline
Spring 2013 \newline}
\bf{For information about citing these materials or our Terms of Use, visit:} \textcolor{blue}{http://ocw.mit.edu/terms.}
\end{document}